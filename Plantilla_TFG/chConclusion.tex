This chapter revolves around bringing a conclusion to the Bachelor`s Thesis. To do so, further work that could improve the results and strengthen its key points is discussed. The Conclusion section includes final thoughts about the Bachelor's thesis and wraps it up by giving and overview of the performed work. 

\section{Further Work}

The possible improvements that can be done to the experiments would be to test the Entropy Triangle on more datasets to further validate our results. We could do so by using some other famous datasets, such as Wine or Boston Housing. It could be interesting to test different transformations such as ICA, but using other Autoencoder structures would be the ideal way to test the ITL. As stated, there is plethora of work available performed with different types of Autoencoder, which could strengthen the validation of the Entropy Triangle by providing more information about how the compression of data is done by them. \par

The code for the experiments was written using R, so an improvement that could be made would be to translate the vignettes into another programming language, preferably Python since it is one of the preferred tools to program in the data science field. It is also free and available worldwide. \par

\section{Conclusion}

The aim of this Bachelor's Thesis was to offer reliable information about a tool available for free in the Internet and how it could be used in different environments to test the reliability of experiments. This tool is easy to use and by following the processes mentioned throughout the thesis they would be easy to mimic. \par

After the problem presented has been tackled, the solution presented was proven viable by using multiple examples and analyzing the results regarded from them. It was also seen that different features of the datasets (such as balancing) affect the classification of their labels. In addition to that, it was also demonstrated that the accuracy paradox is indeed an issue in the data analytics field.

The results presented were consistent with the assertions made before starting them:
\begin{itemize}
	\item The Iris dataset resulted in being an easy dataset to analyze and the balancing of its observations allowed the classifiers to overall perform good.
	
	\item Ionosphere was heavily unbalanced, so the wrong predictions and thus the bad results found on the ET were expected
	
	\item MNIST performed the best when using the ET, and PCA was also proven to outperform the Autoencoder in under specific circumstances.
	
\end{itemize}

Improvements such as adding tests and experiments as well as translating the code into another language were also mentioned as a possible way to strengthen the findings in this Bachelor's Thesis. \par

To sum up, the objectives mentioned at the beginning of the thesis were met, and reliable experiments and information was introduced in order to help researchers around the world improve their methods and, by doing so, expand our knowledge.