%recipe from
% https://tex.stackexchange.com/questions/73074/creating-abbreviations-list-with-manual-entries?rq=1
% Put this in the prelude
%\usepackage{longtable}
%
%\newcommand\nomenclature[2]{#1 & #2 \\}
%\begin{table}[ht]
\begin{center}
\begin{longtable}{@{}p{1cm}@{}p{\dimexpr\textwidth-1cm\relax}@{}}
\caption{Abbreviations used in the text}
\label{tab:loa}\\
\nomenclature{DNN}{Deep Neural Networks}%
\nomenclature{ET}{Entropy Triangle}%
\nomenclature{ITL}{Information-Theoretic Learning}%
\nomenclature{KNN}{K-Nearest Neighbours}
\nomenclature{MLP}{Multilayer Perceptron}
\nomenclature{PCA}{Principal Component Analysis}
% EDA - Exploratory Data Analysis
% CDA - Confirmatory Data Analysis
% canonical correlation analysis (CCA)
%\nomenclature{}{}%
%\nomenclature{$a$}{The number of angels per unit area}%
%\nomenclature{$N$}{The number of angels per needle point}%
%\nomenclature{$A$}{The area of the needle point}%
%\nomenclature{$\sigma$}{The total mass of angels per unit area}%
%\nomenclature{$m$}{The mass of one angel}
\end{longtable}
\end{center}
%\end{table}

