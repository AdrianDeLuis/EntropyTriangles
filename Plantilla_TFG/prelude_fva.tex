% Recipe for todos from http://tex.stackexchange.com/questions/9796/how-to-add-todo-notes
% \todo
\usepackage{xargs}                      % Use more than one optional parameter in a new commands
%\usepackage[pdftex,dvipsnames]{xcolor}  % Coloured text etc.
\usepackage[colorinlistoftodos,prependcaption,textsize=tiny]{todonotes}
\newcommandx{\unsure}[2][1=]{\todo[linecolor=red,backgroundcolor=red!25,bordercolor=red,#1]{#2}}
\newcommandx{\change}[2][1=]{\todo[linecolor=blue,backgroundcolor=blue!25,bordercolor=blue,#1]{#2}}
\newcommandx{\info}[2][1=]{\todo[linecolor=OliveGreen,backgroundcolor=OliveGreen!25,bordercolor=OliveGreen,#1]{#2}}
\newcommandx{\improvement}[2][1=]{\todo[linecolor=Plum,backgroundcolor=Plum!25,bordercolor=Plum,#1]{#2}}
\newcommandx{\thiswillnotshow}[2][1=]{\todo[disable,#1]{#2}}

% FVA: abbreviations as commands
\RequirePackage{xspace}
\newcommand{\bth}{B.~Th.\xspace}

% FVA: abbreviations in a list.
%recipe from
% https://tex.stackexchange.com/questions/73074/creating-abbreviations-list-with-manual-entries?rq=1
% Put this in the prelude
%\usepackage{longtable}
%
%\newcommand\nomenclature[2]{#1 & #2 \\}
% Then create a list like:
%\begin{longtable}{@{}p{1cm}@{}p{\dimexpr\textwidth-1cm\relax}@{}}
%\nomenclature{$a$}{The number of angels per unit area}%
%\nomenclature{$N$}{The number of angels per needle point}%
%\nomenclature{$A$}{The area of the needle point}%
%\nomenclature{$\sigma$}{The total mass of angels per unit area}%
%\nomenclature{$m$}{The mass of one angel}
%\end{longtable}

